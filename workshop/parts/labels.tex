\documentclass[a4paper,times]{article}

\usepackage{times}

\newcommand{\smalllabel}[1]{
\fbox{
\vbox to 10mm {
\hbox to 3.5cm{
\parbox[c]{3.5cm}{
\scriptsize{
#1}}}\vfil}}}


\newcommand{\tlabel}[1]{
#1\\\\
}


\begin{document}


\newcommand{\Xmax}[3]{$\mathrm{#1}_\textrm{max}$=$#3\mathrm{#2}$}
\newcommand{\Pmax}[1]{\Xmax{P}{W}{#1}}
\newcommand{\Imax}[1]{\Xmax{I}{A}{#1}}
\newcommand{\Umax}[1]{\Xmax{U}{V}{#1}}


\newcommand{\resistor}[7]{
\smalllabel{
RESISTOR #1 \textbf{R=#2$\Omega$}

\Pmax{#3} #6

\Umax{#4} \Imax{#5}

#7
}
}


\newcommand{\dipsocket}[3]{
\smalllabel{
#1 SOCKET #2''
\\
\\

#3
}
}

\newcommand{\zener}[6]{
\smalllabel{
ZENER DIODE \textbf{#1{}V} #2

#3 \Pmax{#4}

\Imax{#5}

#6
}
}

\newcommand{\lomexline}[3]{
L(#1) \hfill #3 pcs \hfill #2 CHF
}

\newcommand{\cercap}[5]{
\smalllabel{
\noindent{}CER. CAPACITOR 

\textbf{#1} #3

\Umax{#2} size #4

#5
}
}

\newcommand{\elco}[5]{
\smalllabel{
\noindent{}ELCO \textbf{#1} #3

\Umax{#2} size #4

Take care for polarity!

#5
}
}

\newcommand{\crystal}[2]{
\smalllabel{
CRYSTAL OSCILLATOR

\textbf{#1}
\\

#2
}
}


\newcommand{\unknown}[2]{
\smalllabel{
#1

#2
}
}

\newcommand{\filllabel}{
\smalllabel{
filler

\lomexline{00-00-00}{0.00}{n}
}
}



resistors:\\
\resistor{0207}{1}{0.50}{0.707}{0.71}{02-10-67}{0.08}
\resistor{0207}{10}{0.50}{2.24}{0.22}{02-10-68}{0.08}
\resistor{0207}{100}{0.50}{7.07}{71\mathrm{m}}{02-10-69}{0.08}
\resistor{0207}{1}{0.60}{0.775}{0.77}{02-00-71}{0.07}
\resistor{0207}{10}{0.60}{2.45}{0.24}{02-00-95}{0.06}
\resistor{0207}{5.1}{0.60}{1.75}{0.34}{02-00-88}{0.07}
\resistor{0207}{20}{0.60}{3.46}{0.17}{02-01-02}{0.06}
\resistor{0207}{33}{0.60}{4.45}{0.13}{02-01-07}{0.06}
\resistor{0207}{51}{0.60}{5.53}{0.11}{02-01-12}{0.06}
\resistor{0207}{100}{0.60}{7.75}{77\mathrm{m}}{02-01-19}{0.06}
\resistor{0207}{160}{0.60}{9.8}{61\mathrm{m}}{02-01-24}{0.06}
\resistor{0207}{330}{0.60}{14.1}{43\mathrm{m}}{02-01-31}{0.06}
\resistor{0207}{510}{0.60}{17.5}{34\mathrm{m}}{02-01-36}{0.07}
\resistor{0207}{820}{0.60}{22.2}{27\mathrm{m}}{02-01-41}{0.07}
\resistor{0207}{1k}{0.60}{24.5}{24\mathrm{m}}{02-01-43}{0.07}
\resistor{0207}{3k}{0.60}{42.4}{14\mathrm{m}}{02-01-55}{0.07}
\resistor{0207}{5k}{0.60}{54.8}{11\mathrm{m}}{02-01-60}{0.06}
\resistor{0207}{8k}{0.60}{69.3}{8.7\mathrm{m}}{02-01-65}{0.07}
\resistor{0204}{10k}{0.40}{63.2}{6.3\mathrm{m}}{02-06-83}{0.06}
\resistor{0204}{20k}{0.40}{89.4}{4.5\mathrm{m}}{02-08-28}{0.07}
\resistor{0204}{33k}{0.40}{115}{3.5\mathrm{m}}{02-08-32}{0.07}
\resistor{0204}{47k}{0.40}{137}{2.9\mathrm{m}}{02-08-36}{0.06}
\resistor{0204}{68k}{0.40}{165}{2.4\mathrm{m}}{02-08-40}{0.07}
\resistor{0204}{82k}{0.40}{181}{2.2\mathrm{m}}{02-08-42}{0.07}
\resistor{0204}{100k}{0.40}{200}{2\mathrm{m}}{02-06-85}{0.07}
\resistor{0204}{270k}{0.40}{329}{1.2\mathrm{m}}{02-08-53}{0.06}
\resistor{0204}{680k}{0.40}{522}{0.77\mathrm{m}}{02-08-63}{0.07}
\resistor{0204}{1M}{0.40}{632}{0.63\mathrm{m}}{02-08-67}{0.07}
\resistor{0204}{5.1M}{0.40}{1.43e+03}{0.28\mathrm{m}}{02-08-76}{0.07}
\resistor{0207}{1k}{0.60}{24.5}{24\mathrm{m}}{02-01-47}{0.07}
\resistor{0207}{15k}{0.60}{94.9}{6.3\mathrm{m}}{02-01-71}{0.07}
\resistor{0414}{0.1}{2.00}{0.447}{4.5}{09-00-01}{0.11}
\resistor{0414}{0.2}{2.00}{0.632}{3.2}{09-00-08}{0.11}
\resistor{0414}{1.1}{2.00}{1.48}{1.3}{09-00-26}{0.09}
\resistor{0414}{2}{2.00}{2}{1}{09-00-32}{0.09}
\resistor{0414}{5.1}{2.00}{3.19}{0.63}{09-00-41}{0.09}
\resistor{0414}{10}{2.00}{4.47}{0.45}{09-00-48}{0.09}
\resistor{0414}{100}{2.00}{14.1}{0.14}{09-00-72}{0.09}
%



\include{other}%

unknown:\\
\unknown{DIODE 1N4148 \\ package: DO-35 \\ 0.15A 100V switch}{\lomexline{16-01-04}{0.05}{50}}
\unknown{DIODE 1N4007 \\ package: DO-41 \\ 1A 1000V rectifier}{\lomexline{16-00-26}{0.10}{50}}
\unknown{SCHOTTKY DIODE 1N5819 \\ package: DO-41 \\ 1A 40V rectifier}{\lomexline{16-02-71}{0.10}{25}}
\unknown{RECTIFIER BRIDGE DB107G \\ package: DIP \\ 1A 1000V glass passivated}{\lomexline{20-00-97}{0.20}{10}}
\unknown{NPN universal transistor \\ BC337-40TA package: TO-92 \\ 50V 0.8A 0.63W}{\lomexline{21-07-97}{0.10}{25}}
\unknown{PNP universal transistor \\ BC32740BU package: TO-92 \\ 60V 0.8A 0.625W}{\lomexline{21-07-19}{0.10}{25}}
\unknown{PNP darlington power transistor \\ TIP127 package: TO-220 \\ 100V 5A 75W needs cooling pad!}{\lomexline{21-09-21}{0.30}{8}}
\unknown{NPN darlington power transistor \\ TIP122 package: TO-220 \\ 100V 5A 75W needs cooling pad!}{\lomexline{21-07-49}{0.35}{8}}
\unknown{N-ch J-FET switch J107 \\ $\mathrm{V}_{\mathrm{GS}_\textrm{off}}$=4.5V \Imax{0.1} \\ R=8$\Omega$ TO-92}{\lomexline{21-05-31}{0.25}{10}}
\unknown{N-ch MOSFET switch \\ BS170 package: TO-92 \\ 60V 0.5A 0.83W}{\lomexline{21-01-32}{0.15}{20}}
\unknown{N-ch power MOSFET \\ STP16NE06 R=0.08$\Omega$ TO-220 \\ 60V 16A 45W needs cooling pad!}{\lomexline{21-08-94}{0.35}{20}}
\unknown{P-ch J-FET switch 2N5460 \\ \Pmax{0.31} package: TO-92 \\ \Umax{40} \Imax{10\mathrm{m}} }{\lomexline{21-01-67}{0.25}{10}}
\unknown{P-ch MOSFET switch \\ TP2020L package: TO-92 \\ 200V 0.5A 0.8W}{\lomexline{21-05-10}{0.35}{6}}
\unknown{P-ch power MOSFET \\ 2SJ141 R=0.2$\Omega$ TO-220-F \\ 60V 13A 35W needs cooling pad!}{\lomexline{21-09-77}{0.40}{20}}
\unknown{LINEAR POTENTIOMETER \\ R=10k$\Omega$}{\lomexline{06-01-80}{0.40}{10}}
\unknown{D-SUB CONNECTOR \\ 9 pin female \\ }{\lomexline{43-00-67}{0.25}{8}}
\unknown{D-SUB CONNECTOR \\ 9 pin male \\ }{\lomexline{43-00-68}{0.20}{8}}
\unknown{D-SUB CONNECTOR HOUSING \\ for serial or VGA port \\ }{\lomexline{43-06-24}{0.20}{16}}
\unknown{HD-DSUB CONNECTOR \\ 15 pin male (VGA)}{\lomexline{43-01-03}{0.35}{4}}
\unknown{MINI-JACK 3.5mm CONN. \\ free hanging, female, stereo \\ }{\lomexline{43-05-05}{0.35}{6}}
\unknown{MINI-JACK 3.5mm PLUG \\ male, stereo \\ }{\lomexline{43-05-15}{0.25}{6}}
\unknown{STD. VOLTAGE REGULATOR \\ L7805CV, fixed 5V \Imax{1} \\ TO-220, might need cooling}{\lomexline{32-02-93}{0.30}{10}}
\unknown{STD. VOLTAGE REGULATOR \\ L7812CV, fixed +12V \Imax{1} \\ TO-220, might need cooling}{\lomexline{32-02-28}{0.30}{5}}
\unknown{LDO VOLTAGE REGULATOR \\ LD1117V33 fix. 3.3V \Imax{0.8} \\ TO-220, might need cooling}{\lomexline{32-18-60}{0.35}{10}}
\unknown{ADJ. VOLTAGE REGULATOR \\ LM317TG (ON semi) \\ \Imax{1.5} TO-220}{\lomexline{32-03-95}{0.35}{10}}
\unknown{Super bright LED, RED \\ \O=3 mm  L-934SRD-E (KIN) \\ diffuse, 250mcd, 640nm, $60^\circ$}{\lomexline{34-04-69}{0.10}{50}}
\unknown{Super bright LED, GREEN \\ \O=3 mm L-934SGD (KIN) \\ diffuse, 40mcd, 568nm, $60^\circ$}{\lomexline{34-01-09}{0.10}{50}}
\unknown{LED, YELLOW \\ \O=3mm L-934YDLK (KIN)\\ diffuse, 32mcd, 588nm, $60^\circ$}{\lomexline{34-05-20}{0.10}{50}}
\unknown{Super bright RGB LED \O=5mm \\ HB5-40AOR AGC ABC C (HYA) \\ w.clear $12^\circ$}{\lomexline{34-05-96}{0.70}{10}}
\unknown{FEMALE HEADER \\ 40-pin, 2.54mm pitch \\ }{\lomexline{43-03-71}{0.35}{10}}
\unknown{MALE HEADER \\ 40-pin (can be broken), 2.54mm pitch}{\lomexline{43-00-73}{0.15}{15}}
\unknown{OPERATIONAL AMPLIFIER \\ UA741CP (TI) \\ package: DIP-8}{\lomexline{32-03-36}{0.25}{10}}
\unknown{DC power connector DCS-5521-3 \\ female 5.5/2.1/9.5mm \\ free hanging}{\lomexline{43-11-15}{0.90}{3}}
\unknown{DC power connector DCS-5521-5 \\ female 5.5/2.1/9.5mm \\ for PCB, $90^\circ$}{\lomexline{43-07-67}{0.20}{20}}
\unknown{OPTOCOUPLER \\ LTV817M-V (LIT) package: DIP-4 \\ TRANS-OUT 50-600\% 5kV}{\lomexline{38-03-72}{0.20}{10}}
\unknown{RUBBER FOOT \\ D=6x3mm \\ adhesive (3M)}{\lomexline{47-02-17}{0.05}{144}}
\unknown{PUSHBUTTON momentary \\ DPDT \\ 6-pin, 8x8mm}{\lomexline{45-03-18}{0.20}{10}}
\unknown{MEASUREMENT CLIP \\ with banana receptacle \\ black/red}{\lomexline{43-06-29}{0.20}{8}}
\unknown{MEASUREMENT CLIP \\ with banana receptacle \\ black/red insulated}{\lomexline{43-14-97}{0.50}{4}}
\unknown{BANANA PLUG \\ plastic housing \\ green/white/blue}{\lomexline{43-16-77}{0.50}{6}}
\smalllabel{}
\smalllabel{}
\smalllabel{}
\smalllabel{}
%


\end{document}
